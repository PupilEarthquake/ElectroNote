\documentclass{mynote}
\title{Exercise5}
\author{202005100214}
\date\today

\begin{document}
\maketitle
\begin{exercise}{4.2}
一平面波以$\theta = 45^{\circ}$从真空入射到$\varepsilon_r= 2$的介质,电场强度垂直于入射面,求反射系数和折射系数。
\end{exercise}
\begin{solution}
根据定义
\begin{gather*}
    R = \left| \frac{E'}{E} \right|^2 \\
    T = \left| \frac{E''}{E} \right|^2
\end{gather*}
在$E$垂直于入射面时还有$\dfrac{E'}{E} = \dfrac{\sqrt{\varepsilon_1} \cos \theta - \sqrt{\varepsilon_2}\cos \theta''}{\sqrt{\varepsilon_1} \cos \theta + \sqrt{\varepsilon_2}\cos \theta''}$,$\dfrac{E''}{E} = \dfrac{2 \sqrt{\varepsilon_1} \cos \theta}{\sqrt{\varepsilon_1} \cos \theta + \sqrt{\varepsilon_2} \cos \theta''}$。

由$\dfrac{\sin \theta}{\sin \theta''} = \sqrt{\dfrac{\varepsilon_2}{\varepsilon_1}} = \sqrt{2}$得$\sin \theta'' = \dfrac{1}{2}$,因此$\sin \theta = \cos \theta = \dfrac{\sqrt{2}}{2},\; \cos \theta'' = \dfrac{\sqrt{3}}{2}$,还有$\varepsilon_1 = \varepsilon_0,\; \varepsilon_2 = 2\varepsilon_0$。根据这些求出$R = \dfrac{2 - \sqrt{3}}{2 + \sqrt{3}},\; T = \dfrac{2\sqrt{3}}{2 + \sqrt{3}}$
\end{solution}








\begin{exercise}{4.3}
可见光由水入射到空气,入射角为$60^{\circ}$,证明会发生全反射,求折射波沿表面传播的像速度和透入空气的深度。$\lambda_0 = 6.28\times 10^{-5},\; n=1.33$.
\end{exercise}
\begin{solution}
% \textcolor{red}{?}
$\sin \theta  = \dfrac{\sqrt{3}}{2} > n$,所以会发生全反射。

由$k''\sin \theta'' = k''_x = k_x = k \sin \theta$得$k'' = kn$,$v = \dfrac{\omega ''}{k ''} = \dfrac{\omega}{nk}$,又$\omega = ck$,故$v = \dfrac{c}{n}$.

\textcolor{red}{??????????}
\end{solution}





\begin{exercise}{4.6}
平面电磁波垂直入射到金属表面,证明金属内部电磁波全部转化为焦耳热。
\end{exercise}
\begin{solution}
$\bm{S} = \bm{E} \times \bm{H},\; \omega = \dfrac{1}{2} (\bm{E} \cdot \bm{D} + \bm{H} \cdot \bm{B})$。如果$f(t) = f_0 e^{-i \omega t},\; g(t) = g_0 e^{-i\omega t + i \phi}$,则其平均值定义为$\overline{fg} = \dfrac{1}{2} \textrm{Re} \left[ f^*g \right]$。

该问中$\bm{E} = \bm{E}_0 e^{-\alpha z} e^{i(\beta z - \omega t)}$,$\bm{H} = \dfrac{1}{\omega \mu} \bm{k} \times \bm{E} = \dfrac{1}{\omega \mu} (\alpha i + \beta) \bm{e}_k \times \bm{E}$,得到
\begin{align*}
\bm{S} &= \bm{E} \times \bm{H} \\
&= \bm{E} \times \left[ \frac{1}{\omega \mu} (\alpha i + \beta) \bm{e}_k \times \bm{E} \right]\\
&= (\alpha i + \beta) e^{i (\beta z - \omega t)} \dfrac{E_0^2}{\omega \mu} e^{-2\alpha z} e^{i(\beta z - \omega t)} \hbm{e}_z
\end{align*}
\[
\overline{S} = \frac{1}{2} \textrm{Re} \left[ (\alpha i + \beta) e^{i (\beta z - \omega t)} \dfrac{E_0^2}{\omega \mu} e^{-2\alpha z} e^{-i(\beta z - \omega t)} \right] = \frac{\beta E_0^2}{2\omega \mu} e^{-\alpha z}
\]
热功率密度$ p = \bm{J} \cdot \bm{E} = \sigma E^2 =  \sigma e^{i (\beta z - \omega t)} \dfrac{E_0^2}{\omega \mu} e^{-2\alpha z} e^{i(\beta z - \omega t)}$,得
\[
\overline{p} = \frac{1}{2} \sigma E_0^2 e^{-2 \alpha z}    
\]
单位面积的热功率
\[
P = \int_0^{+\infty} \frac{1}{2} \sigma E_0^2 e^{-2 \alpha z} \dl z = \frac{1}{2} \sqrt{\frac{\sigma}{2\omega \mu}} E_0^2
\]
单位时间进入导体的能流
\[
\overline{S}|_{z=0} =  \frac{1}{2} \sqrt{\frac{\sigma}{2\omega \mu}} E_0^2
\]
\end{solution}








\begin{exercise}{4.9}
无限长矩形导波管,在$z=0$处被一块垂直插入的理想导体平板完全封闭,求$z=-\infty$到$z=0$这段可能存在的波模。
\end{exercise}
\begin{solution}
导波管内满足方程
\[
\left\{
    \begin{aligned}
        & \nabla^2 \bm{E} + k^2 \bm{E} = 0 \\
        & \nabla \cdot \bm{E} = 0 & (\textrm{每一处}) \\
        & \hbm{e}_n \times \bm{E} = 0 & (\textrm{边界处})\\
    \end{aligned} 
\right.    
\]
后两个条件又可以推出
\[
\left\{
    \begin{aligned}
        & E^y = E^z = \pp{E^x}{x} = 0 & (x = 0,\;a) \\
        & E^x = E^z = \pp{E^y}{y} = 0 & (y = 0,\; a) \\
        & E^x = E^y = \pp{E^z}{z} = 0 & (z = 0)
    \end{aligned} 
\right.    
\]
方程$\nabla^2 E + k^2 E = 0$的通解为
\[
E(x,y,z) = (C_x \cos k_x x + D_x \sin k_x x)(C_y \cos k_y y + D_y \sin k_y y)(C_z \cos k_z z + D_z \sin k_z z)        
\]
由$E^x = 0 \;(y=0,\; a,\; z=0)$得到$C_y = 0,\; k_y = \dfrac{n \pi}{a},\; k_x = \dfrac{m \pi}{a},\; C_z = 0$,由$\left. \pp{E^x}{x} \right|_{y=0,\; a} = 0$得$D_x = 0$。可以写出
\[
E^x = A_1 \cos \frac{m\pi}{a}x \sin \frac{n\pi}{a} y \sin k_z z    
\]
同样的方法求出
\begin{gather*}
    E^y = A_2 \sin \frac{m\pi}{a}x \cos \frac{n\pi}{a} y \sin k_z z \\
    E^z = A_3 \sin \frac{m\pi}{a}x \sin \frac{n\pi}{a} y \cos k_z z    
\end{gather*}

要求$k_x^2 + k_y^2 + k_z^2 = \omega^2 \mu \varepsilon \Rightarrow k_z = \sqrt{\omega^2 \mu \varepsilon - \dfrac{m^2 \pi^2}{a^2} - \dfrac{n^2 \pi^2}{a^2}}$,以及$k_x A_1 + k_y A_2 + k_z A_3 = 0$。
\end{solution}








\begin{exercise}{4.11}
写出矩形导波管内磁场$\bm{H}$满足的方程及边界条件。
\end{exercise}
\begin{solution}
$\nabla \times \bm{H} = \pp{\bm{D}}{t},\; \nabla \times \bm{E} = -\pp{\bm{B}}{t}$,所以
\[
-\nabla^2 \bm{H} = \nabla(\nabla \cdot \bm{H}) - \nabla^2 \bm{H} =  \nabla \times (\nabla \times \bm{H}) = \omega^2 \mu \varepsilon \bm{H}    
\]
得方程
\[
\left\{
    \begin{aligned}
        & \nabla^2 \bm{H} + k^2 \bm{H} = 0 \\
        & \nabla \cdot \bm{H} = 0 & (\textrm{每一处}) \\
        & \hbm{e}_n \times \bm{H} = \bm{a} & (\textrm{边界处})\\
    \end{aligned} 
\right.    
\]
\end{solution}







\begin{exercise}{4.12}
论证矩形导波管不存在$\textrm{TM}_{m0},\; \textrm{TM}_{0n}$波。
\end{exercise}
\begin{solution}
导波管内满足方程
\[
\left\{
    \begin{aligned}
        & \nabla^2 \bm{E} + k^2 \bm{E}= 0 \\
        & \nabla \cdot \bm{E} = 0 & (\textrm{每一处}) \\
        & \hbm{e}_n \times \bm{E} = 0 & (\textrm{边界处})\\
    \end{aligned} 
\right.    
\]
在$z=0$面没有约束时,得通解
\[
\left\{
    \begin{aligned}
        & E^x = A_1 \cos k_x x \sin k_y y e^{ik_z z} \\
        & E^y = A_2 \sin x_x x \cos k_y y e^{ik_z z} \\
        & E^z = A_3 \sin k_x x \sin k_y y e^{ik_z z} 
    \end{aligned} 
\right.    
\]
由$\bm{H} = -\dfrac{i}{\omega \mu} \nabla \times \bm{E}$
\begin{align*}
    H^x &= -\frac{i}{\omega \mu} \epsilon^{ij1} \partial_i E_j\\
    &= -\frac{i}{\omega \mu} \left[ A_1 k_y \sin k_x x \cos k_y y e^{ik_z z} - ik_z zA_2 \sin k_x x \cos k_y y e^{ik_z z} \right]
\end{align*}
同理有
\begin{gather*}
    H^y = -\frac{i}{\omega \mu} \left[ i A_1 k_z - A_3 k_x \right] \cos k_x x \sin k_y y e^{ik_z z} \\
    H^z = -\frac{i}{\omega \mu} \left[ A_2 k_x - A_1 k_y \right] \cos k_x x \cos k_y y e^{ik_z z}
\end{gather*}
需要$k^2 = \omega^2 \mu \varepsilon = k_x^2 + k_y^2 + k_z^2,\; k_x = \dfrac{m \pi}{a},\; k_y = \dfrac{n \pi}{b}$,$A_1k_x + A_2k_y + ik_z A_3 = 0$。

TM波$H^z = 0 \Rightarrow A_2k_x - A_1 k_y = 0$

$\textrm{TM}_{m0}$波:$n=0 \Rightarrow k_y = 0,\; A_2 = 0 \Rightarrow H^x = H^y = H^z = 0$。


$\textrm{TM}_{0n}$波:$m=0 \Rightarrow k_x = 0, A_1 = 0 \Rightarrow H^x = H^y = H^z = 0$。


\end{solution}










\begin{exercise}{4.13}
频率为$30 \times 10^9 \textrm{Hz}$的微波,在$0.7 \times 0.4 \textrm{cm}$的矩形导波管内能以什么波模传播,在$0.7 \times 0.6 \textrm{cm}$呢。
\end{exercise}
\begin{solution}
截至频率$\omega_{cmn} = \dfrac{\pi}{\mu \varepsilon} \sqrt{\dfrac{m^2}{a^2} + \dfrac{n^2}{b^2}} = \pi c \sqrt{\dfrac{m^2}{a^2} + \dfrac{n^2}{b^2}} $
% ,波长$\lambda_{cmn} = \dfrac{2\pi c}{\omega_{cmn}} = 2\dfrac{1}{\sqrt{\dfrac{m^2}{a^2} + \dfrac{n^2}{b^2}}}$。

可计算出得一种情况可以传播$\textrm{TE}_{00 \sim 55},\;\textrm{TE}_{65} $,第二种情况可以传播$\textrm{TE}_{00 \sim 99},\; \textrm{TE}_{9,10},\; \textrm{TE}_{10, 9}$
\end{solution}






















\end{document}