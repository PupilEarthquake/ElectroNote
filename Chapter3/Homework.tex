\documentclass{mynote}
\title{Homework4}
\author{2020051001214}
\date\today

\begin{document}
\maketitle




\begin{exercise}{3.3}
    无穷长线电流$I$沿$z$轴流动,$z<0$的区域磁导率为$\mu$,$z>0$的空间磁导率为$\mu_0$,求磁感应强度$\bm{B}$,求磁化电流分布。
\end{exercise}
\begin{solution}
设$z>0$区域的失势为$\bm{A}_1$,$z<0$为$\bm{A}_2$,在挖去线电流的空间中有方程组
\[
\left\{
    \begin{aligned}
        & \nabla^2 \bm{A}_1 = 0\\
        & \nabla^2 \bm{A}_2 = 0 \\
        & \hbm{e}_n \cdot (\bm{B}_2 - \bm{B}_1) |_{z=0}= 0 \\
        & \hbm{e}_n \times (\dfrac{1}{\mu} \bm{B}_2 - \dfrac{1}{\mu_0} \bm{B}_2) |_{z=0} = \bm{a}_M
    \end{aligned} 
\right.    
\]
在柱坐标下考察方程$ \nabla^2 \bm{A}_1 = 0$,由于对称性$\bm{A}_1$只与$R$有关,即$\dfrac{1}{R} \pp{}{R} \left( R \pp{A_1^{i}}{R} \right) = 0$,简写为$R^2(A_1^{i})'' + R(A_1^{i}) ' = 0 $,此为欧拉方程,(令$R = e^t$)解得$A_1^i = a_1^i \ln R + b_1^i$,所以$\bm{A}_1 = \bm{a}_1 \ln R + \bm{b}_1,\; \bm{A}_2 = \bm{a}_2 \ln R + \bm{b}_2$。
\[
\bm{B}_1 = \nabla \times (\ln R \bm{a}_1) = \dfrac{1}{R} \left| \begin{matrix}
    \hbm{e}_R & R\hbm{e}_{\phi} & \hbm{e}_z \\
    \partial_R & \partial_{\phi} & \partial_{z} \\
    a_1^R \ln R &R a_1^{\phi} \ln R & a_1^z \ln R
\end{matrix} \right|    = \frac{-a_1^{z}}{R} \hbm{e}_{\phi} + a_1^{\phi} \left( \frac{\ln R}{R} + \frac{1}{R} \right) \hbm{e}_z
\]
根据常识可知这种情况下$\bm{B}_1$没有$z$方向分量,所以$a_1^{\phi} = 0$,此时可以满足约束条件$\hbm{e}_n \cdot (\bm{B}_2 - \bm{B}_1) = 0$。另一方面,用$\bm{B}$沿与$z$垂直,圆心在$z$上且半径为$R$的取线上积分,得$\displaystyle\oint_{r=R} \frac{-a_1^{z}}{R} \hbm{e}_{\phi} \cdot \dl \bm{l} = \mu_0 I$,解出$a_1^z = -\dfrac{\mu_0 I}{2 \pi}$,最终得到$\bm{B}_1 = \dfrac{\mu_0 I}{2 \pi R} \hbm{e}_{\phi},\; \bm{B}_2 = \dfrac{\mu I}{2 \pi R} \hbm{e}_{\phi}$.


介质分界面处满足$\hbm{e}_n \times (\dfrac{1}{\mu} \bm{B}_2 - \dfrac{1}{\mu_0} \bm{B}_1) = \bm{a}_M = 0$

磁化电流密度满足$\bm{J}_M = \nabla \times \bm{M}$,而$\bm{M} = \dfrac{\bm{B}}{\mu_0} - \dfrac{\bm{B}}{\mu}$,所以$\bm{J}_1 = 0$
\[
\bm{J}_2 = \left( \frac{1}{\mu_0} - \frac{1}{\mu} \right) \nabla \times \bm{B}_2 = \left( \frac{1}{\mu} - \frac{1}{\mu_0} \right)\frac{\mu I}{2\pi R^3} \hbm{e}_z    
\]

\end{solution}








\begin{exercise}{3.4}
    $x<0$处磁导率为$\mu$,$x>0$处为$\mu_0$,电流和上问一样,求磁感应强度和磁化电流强度。
\end{exercise}
\begin{solution}
\[
    \left\{
        \begin{aligned}
            & \nabla^2 \bm{A}_1 = 0\\
            & \nabla^2 \bm{A}_2 = 0 \\
            & \hbm{e}_n \cdot (\bm{B}_2 - \bm{B}_1) |_{\phi = 0,\; \pi}= 0 \\
            & \hbm{e}_n \times (\dfrac{1}{\mu}_0 \bm{B}_2 - \dfrac{1}{\mu} \bm{B}_1) |_{\phi = 0,\; \pi}=\bm{a}_M
        \end{aligned} 
    \right.    
 \]
 $\bm{A}$与$z$无关,所以不妨直接让$\bm{r}$在$x-y$平面上。在已知空间电流分布时可用$\bm{A}  = \dfrac{\mu}{4 \pi} \displaystyle\int_{\mathcal{V}} \dfrac{\bm{J} \dl \tau '}{\imath}$求解。此题中$\bm{J}$仅在$z$轴上有效,因此
 \[
 \bm{A} = \frac{\mu}{4\pi} \int_{\mathcal{P}} \int_{\mathcal{S}} \dfrac{\bm{J} \dl \bm{a}' \cdot \dl \bm{l}'}{\imath} = \frac{\mu}{4\pi} \int_{-\infty}^{\infty} \frac{I \dl z'}{\sqrt{R^2 + z'^2}} \hbm{e}_z \Rightarrow \bm{A} = \frac{\mu I}{4 \pi} \ln \frac{1}{R} \hbm{e}_z
 \]
 由$\nabla \times \bm{A} = \bm{B}$求出$\bm{B}_1 = \dfrac{\mu I}{4 \pi R} \hbm{e}_{\phi},\; \bm{B}_2 = \dfrac{\mu_0 I}{4 \pi R} \hbm{e}_{\phi}$。此结果满足约束条件$\hbm{e}_n \cdot (\bm{B}_2 - \bm{B}_1) |_{\phi = 0,\; \pi}= 0$,并且可求出介质分界面处$\bm{a}_M = 0$。

同上一问一样有$\bm{J}_1 =  \left( \dfrac{1}{\mu} - \dfrac{1}{\mu_0} \right)\dfrac{\mu I}{2\pi R^3} \hbm{e}_z ,\; \bm{J}_2 = 0$.

\end{solution}





\begin{exercise}{3.7}
    半径为$a$的无限长圆柱导体上有恒定电流$\bm{J}$,解失势的微分方程,设导体磁导率为$\mu$,导体外磁导率为$\mu_0$。
\end{exercise}
\begin{solution}
设导体内失势为$\bm{A}_1$,导体外为$\bm{A}_2$,满足方程
\[
\left\{
    \begin{aligned}
        & \nabla^2 \bm{A}_1 = -\mu \bm{J} \\
        & \nabla^2 \bm{A}_2 = 0 \\
        % & \bm{A}_1 |_{R=0} < \infty
        % & \bm{A}_2 |_{R \to \infty} < \infty
    \end{aligned} 
\right.    
\]    
矢量场$\bm{J}$在圆柱内任何点处的矢量都只有$z$分量,因此上述方程组归结为一下两种方程$\nabla^2 A = -\mu J,\; \nabla^2 A = 0$。首先求解第二个,$\bm{A}$与$z,\phi$无关,所以其分量也只与$R$有关。从而有
\[
\frac{1}{R} \pp{}{R} \left( R \pp{A}{R} \right) = 0   
\]
也即$R^2 A '' _{RR} + R A '_{R} = 0$,令$R = e^t$,则$t = \ln R$,因此方程化为$、\tilde{A}''_{tt} = 0$,解得$\tilde{A} = at + b \Leftrightarrow A= a\ln R + b$。对于方程
\[
    \frac{1}{R} \pp{}{R} \left( R \pp{A}{R} \right) = -\mu J
\]
按照同样方法化为$\tilde{A}''_{tt} = -\mu J e^{2t}$,通解已经求出,特解为$ce^{2t}$,将$\tilde{A} = at + b + ce^{2t}$带回方程,求出$c=-\dfrac{\mu J}{4}$,所以$A = a\ln R + b -\dfrac{\mu J}{4} R^2$

\end{solution}








\begin{exercise}{3.9}
    将一磁导率$\mu$,半径为$R_0$的球体放入均匀磁场$\bm{H}_0$内,求总磁感应强度$\bm{B}$和诱导磁矩$\bm{m}$。
\end{exercise}
\begin{solution}
以$z$方向为极轴,建立球坐标系。假设$\bm{H}_0$的方向为$z$方向。设球体内的标势为$u_1$,球体外为$u_2$,满足
\[
    \left\{
        \begin{aligned}
            & \nabla^2 u_1 = 0 \\
            & \nabla^2 u_2 = 0 \\
            & u_1|_{R_0} = u_2|_{R_0} \\
            & \mu \pp{u_2}{R} - \mu_0 \pp{u_1}{R} = 0\\
            & \lim_{R\to 0} u_1 < \infty 
        \end{aligned} 
    \right.
\]
在均匀磁场强度中,任选一点为原点,假设此处磁标势为$u_0$,则有$\lim\limits_{R \to \infty}u_2 - u_0 = -\bm{r} \cdot \bm{H}_0$,得到$\lim\limits_{R \to \infty}u_2 = u_0 - R H_0 \cos \theta$,此为另一约束条件。

在轴对称的情况下可以套用通解公式,得出
\begin{gather*}
    u_1 = \sum_{n=0} \left( a_n R^n + \frac{b_n}{R^{n+1}} \right) P_n (\cos \theta) \\
    u_2 = \sum_{n=0} \left( c_n R^n + \frac{d_n}{R^{n+1}} \right) P_n (\cos \theta)
\end{gather*}
带入最后两个约束条件,有
\[
\left\{
    \begin{aligned}
        & u_1 = a_0 + a_1 R P_1(\cos \theta) + \sum_{n=2} a_n R^n P_n (\cos \theta) \\
        & u_2 = u_0 + \frac{d_0}{R} + (-H_0 R + \frac{d_1}{R^2}) P_1(\cos \theta) + \sum_{n=2} \frac{d_n}{R^{n+1}}P_n(\cos \theta)
    \end{aligned}
\right.    
\]
考虑正交性,有$a_n = d_n = 0\;(n \geq 2)$,再带入剩下的约束条件,可的方程组
\[
\left\{
    \begin{aligned}
        & a_0 = u_0 + \frac{d_0}{R_0} \\
        & a_1 R_0 = -H_0 R_0 + \frac{d_1}{R_0^2} \\
        & a_1 \mu_0 = -\mu H_0 - \frac{2\mu d_1}{R_0^3} \\
        & d_0 = 0
    \end{aligned} 
\right.    
\]
解得
\[
\left\{
    \begin{aligned}
        & a_0 = u_0 \\
        & a_1 = \frac{3\mu H_0}{2\mu + \mu_0} \\
        & d_0 = 0 \\
        & d_1 = \frac{3\mu H_0 R_0^3}{2\mu + \mu_0} + H_0 R_0^3
    \end{aligned} 
\right.    
\]
因此有
\begin{gather*}
    u_1 = u_0 + \frac{3\mu H_0}{2\mu + \mu_0} R \cos \theta \\
    u_2 = u_0 -H_0 R \cos \theta + (\frac{3\mu H_0 R_0^3}{2\mu + \mu_0} + H_0 R_0^3) \frac{1}{R^2} \cos \theta
\end{gather*}
由$\bm{B} = -\dfrac{1}{\mu} \nabla u$,(这里$\nabla = \hbm{e}_r \pp{}{R} + \hbm{e}_{\theta} \dfrac{1}{R} \pp{}{\theta} + \hbm{e}_{\phi} \dfrac{1}{R\sin \theta} \pp{}{\phi}$)求出
\begin{gather*}
    \bm{B}_1 = \frac{3 H_0}{2\mu + \mu_0} \left( \sin \theta \hbm{e}_{\theta} - \cos \theta \hbm{e}_{r} \right) \\
    \bm{B}_2 = \left[ -H_0 \cos \theta + (\frac{3\mu H_0 R_0^3}{2\mu + \mu_0} + H_0 R_0^3) \cos \theta \frac{-2}{R^3} \right] \hbm{e}_r + \left[ H_0 R \sin \theta - (\frac{3\mu H_0 R_0^3}{2\mu + \mu_0} + H_0 R_0^3)\frac{1}{R^2} \sin \theta \right] \hbm{e}_{\theta}
\end{gather*}


磁偶极矩形成的磁场形如$\bm{B}_m  = -\dfrac{\mu_0}{4\pi} (\bm{m} \cdot \nabla) \dfrac{\bm{R}}{R^3}$,,,,,,\textcolor{red}{(?)}

\end{solution}








\begin{exercise}{3.10}
    内外半径为$R_1,\; R_2$的空心球,位于均匀外磁场$\bm{H}_0$内,球的磁导率为$\mu$,求空腔内的场$\bm{B}$,讨论$\mu \gg \mu_0$的屏蔽效果。
\end{exercise}
\begin{solution}
设标势由内到外分别为$u_1,\; u_2,\; u_3$,满足如下方程
\[
\left\{
    \begin{aligned}
        & \nabla ^2 u_1 = 0 \\
        & \nabla ^2 u_2 = 0 \\
        & \nabla ^2 u_3 = 0 \\
        & u_1|_{R_1} = u_2|{R_1} \\
        & u_2|_{R_2} = u_3|_{R_2} \\
        & \left. \mu_0 \pp{u_1}{R} \right|_{R_1} = \left. \mu \pp{u_2}{R} \right|_{R_1} \\
        & \left. \mu \pp{u_2}{R} \right|_{R_2} = \left. \mu_0 \pp{u_3}{R} \right|_{R_2} \\
        & u_1(0) < \infty \\
        & \lim_{R \to \infty} u_3 = u_0 - RH_0 \cos \theta
    \end{aligned} 
\right.    
\]
得通解
\begin{gather*}
    u_1 = a_0 + a_1 R \cos \theta + \sum_{n=2} a_n R^n P_n(\cos \theta) \\
    u_2 = b_0 + \frac{c_0}{R} + (b_1 R + \frac{c_1}{R^2})P_1(\cos \theta) + \sum_{n=2} (b_nR^n + \frac{c_n}{R^{n+1}}) P_n(\cos \theta) \\
    u_3 = d_0 + \frac{e_0}{R} + (d_1 R + \frac{e_1}{R^2})P_1(\cos \theta) + \sum_{n=2} (d_nR^n + \frac{e_n}{R^{n+1}}) P_n(\cos \theta) 
\end{gather*}
带入约束条件后可得
\[
\left\{
    \begin{aligned}
        & a_0 = b_0 \\
        & c_0 = 0 \\
        & a_1 R_1 = b_1 R_1 + \frac{c_1}{R_1^2} \\
        & c_n = 0 \\
        & b_n = a_n
    \end{aligned} 
\right.
\quad\left\{
    \begin{aligned}
        & b_0 = -u_0 \\
        & c_0 = e_0 \\
        & b_1 R_2 + \frac{c_1}{R_2^2} = -H_0 R_2 + \frac{e_1}{R_2^2} \\
        & d_n = 0\\
        & b_n = e_n = 0
    \end{aligned} 
\right. 
\quad
\left\{
    \begin{aligned}
        & c_0 = e_0 = 0\\
        & \mu_0 a_1 = \mu (b_1 - \frac{2c_1}{R_1^3}) \\
        & \mu (b_1 - \frac{2c_1}{R_2^3}) = \mu_0 (-H_0 - \frac{2e_1}{R_2^3})
    \end{aligned} 
\right. 
\]
化为线性方程组
\[
\begin{bmatrix}
    \mu_0 & -\mu & \dfrac{2\mu}{R_1^3} & 0 \\
    0 & \mu & -\dfrac{2\mu}{R_2^3} & \dfrac{2\mu_0}{R_2^3} \\
    R_1 & -R_1 & -\dfrac{1}{R_1^2} & 0 \\
    0 & R_2 & \dfrac{1}{R_2^2} & \dfrac{1}{R_2^2} 
\end{bmatrix}    
\begin{bmatrix}
    a_1 \\ b_1 \\ c_1 \\ e_1 
\end{bmatrix}
+ \begin{bmatrix}
    0 \\ \mu_0 H_0 \\ 0 \\ R_2 H_0 
\end{bmatrix}
 = 0
\]
由克拉默法则$x_i = \dfrac{\det \bm{A}_i (\bm{b})}{\det \bm{A}}$可以求出相应的解。这里只考察空腔内的场。$\bm{B}_1 = -\mu_0 \nabla u_1 = -\mu_0 a_1 (\cos \theta \hbm{e}_r - \sin \theta \hbm{e}_{\theta})$,所以只需要考察$a_1$的情况。将$\bm{A},\; \tilde{\bm{A}} = \bm{A}_1(\bm{b})$沿第二列展开,即$\bm{A} = -\mu A_{21} + \mu A_{22} - R_1 A_{23} + R_2 A_{44}$,可以验证代数余子式$A_{21},\;A_{22},\;A_{23},\; A_{24}$均是$\mu$的一次式,故$\det \bm{A}$是$\mu$的二次多项式$\mlP_2(\mu)$,同理$\det \tilde{\bm{A}} = \tilde{\mlP}_2(\mu)$,从而有
\[
\lim_{\mu \to \infty} \frac{\mlP_2(\mu)}{\tilde{\mlP}_2(\mu)} = \Const    
\]
所以说$\mu \gg \mu_0$时空腔中的$\bm{B}$恒为常数。




\end{solution}























\end{document}

